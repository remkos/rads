\documentstyle[a4wide]{article}
\title{DUT/SSR\&T Crossover Minimization File Formats}
\author{Edwin Wisse, Marc Naeije, and Remko Scharroo}
\begin{document}
\maketitle

\begin{table}
\caption{Abridged data records}
\medskip
\begin{tabular}{|rr@{--}lc|p{0.7\textwidth}|}
\hline
field&\multicolumn{2}{c}{bytes}&type&description \\
\hline
\multicolumn{5}{c}{\bf header} \\
\hline
 1 & 1 & 4 & A4 & File descriptor ( = 'aADR' ) \\
 2 & 5 &12 & A8 & Satellite \\
 3 &13 &14 & I2 & Lower longitude boundary (degrees) \\
 4 &15 &16 & I2 & Higher longitude boundary (degrees) \\
 5 &17 &18 & I2 & Lower latitude boundary (degrees) \\
 6 &19 &20 & I2 & Higher latitude boundary (degrees) \\
 7 &21 &24 & I4 & Number of data records \\
\hline
\multicolumn{5}{c}{\bf data records} \\
\hline
 1 & 1 & 4 & I4 & Time in UTC seconds from 1985.0 \\
 2 & 5 & 8 & I4 & idem, microsecond part \\
 3 & 9 &12 & I4 & Latitude in microdegrees \\
 4 &13 &16 & I4 & Longitude in microdegrees \\
 5 &17 &20 & I4 & Orbital altitude in millimeters \\
 6 &21 &22 & I2 & Relative sea height in millimeters \\
 7 &23 &24 & I2 & Sea height sigma in millimeters \\
\hline
\end{tabular}
\end{table}

\begin{table}
\caption{Track file format}
\medskip
\begin{tabular}{|rr@{--}lc|p{0.7\textwidth}|}
\hline
field&\multicolumn{2}{c}{bytes}&type&description \\
\hline
\multicolumn{5}{c}{\bf header} \\
\hline
 1 & 1 & 4 & A4 & File descriptor ( = '@XTB' ) \\
 2 & 5 & 8 & I4 & Number of data records \\
 3 & 9 &12 & I4 & Number of orbit parameters (3 or 5) \\
\hline
\multicolumn{5}{c}{\bf data records} \\
\hline
 1 & 1 & 2 & I2 & Track number \\
 2 & 3 & 4 & I2 & Satellite ID (1=GEOS-3, 2=Seasat, 3=Geosat, 4=ERS--1,
 5=TOPEX, 6=Poseidon, 7=ERS--2) \\
 3 & 5 & 6 & I2 & Number of crossovers on this track \\
 4 & 7 & 8 & I2 & Number of altimeter measurements on this track \\
 5 & 9 &12 & I4 & Inclination in microdegrees \\
 6 &13 &16 & I4 & Start argument of latitude in microdegrees \\
 7 &17 &20 & I4 & Time of the nodal passage in Sec85 \\
 8 &21 &24 & I4 & Longitude of the node in microdegrees \\
 9 &25 &28 & I4 & Start time of the track in Sec85 \\
10 &29 &32 & I4 & Stop time of the track in Sec85 \\
\hline
11--13&33 & 44 & 3*I4 & 3 Orbit parameter values in microns (coefficients
   to a constant, a sine and cosine of 1-cpr, and a sine and cosine of
   2-cpr) \\
14--16&45 & 56 & 3*I4 & Sigmas to the 3 orbit parameters in microns \\
17 & 57 & 58 & B2 & Flag bits: bit 1 = ascending track, bit 2 = short track,
   bit 8 = valid track \\
\hline
11--15&33 & 52 & 5*I4 & 5 Orbit parameter values in microns (coefficients
   to a constant, a sine and cosine of 1-cpr, and a sine and cosine of
   2-cpr) \\
16--20&53 & 72 & 5*I4 & Sigmas to the 5 orbit parameters in microns \\
21 & 73 & 74 & B2 & Flag bits: bit 1 = ascending track, bit 2 = short track,
   bit 8 = valid track \\
\hline
\end{tabular}
\end{table}

\begin{table}
\caption{Altimeter file format}
\medskip
\begin{tabular}{|rr@{--}lc|p{0.7\textwidth}|}
\hline
field&\multicolumn{2}{c}{bytes}&type&description \\
\hline
\multicolumn{5}{c}{\bf header} \\
\hline
 1 & 1 & 4 & A4 & File descriptor ( = '@XAB' ) \\
 2 & 5 & 8 & I4 & Number of data records \\
\hline
\multicolumn{5}{c}{\bf data records} \\
\hline
 1 & 1 & 4 & I4 & Measurement time in Sec85 \\
 2 & 5 & 8 & I4 & Latitude of the measurement in microdegrees \\
 3 & 9 &12 & I4 & Longitude of the measurement in microdegrees \\
 4 &13 &16 & I4 & A priori measurement sea surface height in microns \\
 5 &17 &20 & I4 & A posteriori measurement sea surface height in microns \\
 6 &21 &24 & I4 & Argument of latitude of the measurement in microns \\
 7 &25 &26 & I2 & Measurement sigma in millimeters \\
 8 &27 &28 & I2 & Track number \\
\hline
\end{tabular}
\end{table}

\begin{table}
\caption{Crossover file format}
\medskip
\begin{tabular}{|rr@{--}lc|p{0.7\textwidth}|}
\hline
field&\multicolumn{2}{c}{bytes}&type&description \\
\hline
\multicolumn{5}{c}{\bf header} \\
\hline
 1 & 1 & 4 & A4 & File descriptor ( = '@XXB' ) \\
 2 & 5 & 8 & I4 & Number of data records \\
\hline
\multicolumn{5}{c}{\bf data records} \\
\hline
 1 & 1 & 4 & I4 & Latitude of crossover in microdegrees \\
 2 & 5 & 8 & I4 & Longitude of crossover in microdegrees \\
 3 & 9 &12 & I4 & Time of measurement A in Sec85 \\
 4 &13 &16 & I4 & Time of measurement B in Sec85 \\
 5 &17 &18 & I2 & Track number A \\
 6 &19 &20 & I2 & Track number B \\
 7 &21 &24 & I4 & A priori sea height A in microns \\
 8 &25 &28 & I4 & A priori sea height B in microns \\
 9 &29 &32 & I4 & A posteriori sea height A in microns \\
10 &33 &36 & I4 & A posteriori sea height B in microns \\
11 &36 &40 & I4 & Argument of latitude A in microdegrees \\
12 &41 &44 & I4 & Argument of latitude B in microdegrees \\
13 &45 &46 & I2 & Measurement sigma A in millimeters \\
14 &47 &48 & I2 & Measurement sigma B in millimeters \\
\hline
\end{tabular}
\end{table}

\begin{table}
\caption{Crossover file format (with orbit)}
\medskip
\begin{tabular}{|rr@{--}lc|p{0.7\textwidth}|}
\hline
field&\multicolumn{2}{c}{bytes}&type&description \\
\hline
\multicolumn{5}{c}{\bf header} \\
\hline
 1 & 1 & 4 & A4 & File descriptor ( = '@XXO' ) \\
 2 & 5 & 8 & I4 & Number of data records \\
\hline
\multicolumn{5}{c}{\bf data records} \\
\hline
 1 & 1 & 4 & I4 & Latitude of crossover in microdegrees \\
 2 & 5 & 8 & I4 & Longitude of crossover in microdegrees \\
 3 & 9 &12 & I4 & Time of measurement A in Sec85 \\
 4 &13 &16 & I4 & idem, microsecond part \\
 5 &17 &20 & I4 & Time of measurement B in Sec85 \\
 6 &21 &24 & I4 & idem, microsecond part \\
 7 &25 &26 & I2 & Track number A \\
 8 &27 &28 & I2 & Track number B \\
 9 &29 &32 & I4 & Sea height A in microns \\
10 &33 &36 & I4 & Sea height B in microns \\
11 &37 &40 & I4 & Orbital altitude A in millimeters \\
12 &41 &44 & I4 & Orbital altitude B in millimeters \\
\hline
\end{tabular}
\end{table}

\begin{table}
\caption{Normal point file format}
\medskip
\begin{tabular}{|rr@{--}lc|p{0.7\textwidth}|}
\hline
field&\multicolumn{2}{c}{bytes}&type&description \\
\hline
\multicolumn{5}{c}{\bf header} \\
\hline
 1 & 1 & 4 & A4 & File descriptor ( = '@XGF' or '@TIM') \\
 2 & 5 & 8 & I4 & Number of data records \\
\hline
\multicolumn{5}{c}{\bf data records} \\
\hline
 1 & 1 & 2 & I2 & Satellite ID (negative is descending) \\
 2 & 3 & 4 & I2 & Number of points for this normal point \\
 3 & 5 & 8 & I4 & Latitude of normal point in microdegrees \\
 4 & 9 &12 & I4 & Longitude of normal point in microdegrees \\
 5 &13 &16 & I4 & Mean sea level in microns \\
 6 &17 &18 & I4 & Sea surface variability in millimeters \\
\hline
\multicolumn{5}{c}{\bf time serie records following each data record}\\
\multicolumn{5}{c}{\bf (only for type '@TIM')} \\
\hline
 1 & 1 & 4 & I4 & Measurement epoch in Sec85 \\
 2 & 5 & 8 & I4 & Latitude of measurement in microdegrees \\
 3 & 9 &12 & I4 & Longitude of measurement in microdegrees \\
 4 &13 &16 & I4 & Relative sea level in microns \\
 5 &17 &18 & I4 & Negative order number \\
\hline
\end{tabular}
\end{table}

\end{document}
\end{document}
