\documentclass[a4paper]{article}
\usepackage{a4wide,verbatim,som}
\title{SPLIB \\ A Spectral Analysis Subroutine Library \\
User's Guide and Reference}
\author{Remko Scharroo}
%%%%%%%%%%%%%%%%%%%%%%%%%%%%%%%%%%%%%%%%%%%%%%%%%%%%%%%%%%%%%%%%%%%%%%%%
\def\Input#1{\vbox{\input{#1}}}
\def\SECTION#1{\subsection{#1}}
%%%%%%%%%%%%%%%%%%%%%%%%%%%%%%%%%%%%%%%%%%%%%%%%%%%%%%%%%%%%%%%%%%%%%%%%
\begin{document}
\somfront
\tableofcontents

\section{Introduction}
SPLIB is a subroutine library that contains some handy routines to be
used for spectral analyses, Fourier Transforms, etc.

The SPLIB library has been ported to various systems. Presently it is
only available on the IBM Risk 6000 workstations of the Section Space
Research and Technology and the Convex Supercomputer of the Central
Computing Facility of the Delft University.

The library can be linked to any of your FORTRAN programs by specifying
including the library in your FORTRAN link step, \eg
\begin{verbatim}
f77 example.f -o example /user/altim/lib/rssubs.a
\end{verbatim}

The SPLIB library is courtesy of Remko Scharroo, Delft University of
Technology, Section Space Research \&\ Technology
(E-mail: {\tt Remko.Scharroo@lr.tudelft.nl})

\section{Subroutine Synopsis}
\Input{spcdem}
\Input{spcomp}
\Input{spdftc}
\Input{spdftr}
\input{spectr}
\Input{spfamp}
\Input{spfftc}
\Input{spfftr}
\Input{spfper}
\Input{sprand}
\Input{statis}
\end{document}
