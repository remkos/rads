\\SECTION{CSRTIDE -- Compute tides according to CSR model}
\\begin{verbatim}
subroutine csrtide (utc, lat, lon, tide, tide_lp, tide_load)
use csrmod
real(eightbytereal), intent(in) :: utc, lat, lon
real(eightbytereal), intent(out) :: tide, tide_lp, tide_load

Compute ocean tidal height for given time and location from grids
of ortho weights for one of the CSR models.
This routine is heavily based on the routine CSRTPTIDE by Richard Eanes.

For partical purposes I have changed the input to netCDF grids.
These grids can be found in $ALTIM/data/csr_tides, where $ALTIM is an
environment variable.

To initialize the computation, the function csrinit should be
called first. It allocates the appropriate amount of memory and
loads the grids into memory. To release the memory for further
use, call csrfree.

Longitude and latitude are to be specified in degrees; time in UTC
seconds since 1 Jan 1985. All predicted tides are output in meters.
If the tide is requested in a point where it is not defined, NaN
(Not-a-Number) is returned.

Input arguments:
 utc      : UTC time in seconds since 1 Jan 1985
 lat      : Latitude (degrees)
 lon      : Longitude (degrees)

Output arguments:
 tide     : Predicted short-period tide (m)
 tide_lp  : Predicted long-period tide (m)
 tide_load: Predicted loading effect (m)
\\end{verbatim}

\\SUBSECTION{CSRINIT -- Initialize CSR tide model}
\\begin{verbatim}
subroutine csrinit (name, wantload)
use csrmod
character(*) :: name
logical :: wantload

Allocate memory for CSR tide modeling and read grids into memory.
When WANTLOAD is .TRUE., loading tide grids are loaded and load tide
will be computed. When .FALSE., CSRTIDE will return a zero load tide.

Input arguments:
 name     : Name of the CSR tide model (csr_3.0 or csr_4.0)
 wantload : Specify that load tide has to be computed
\\end{verbatim}

\\SUBSECTION{CSRFREE -- Free up space allocated by CSRINIT}
\\begin{verbatim}
subroutine csrfree
use csrmod

This routine frees up memory allocated by a previous call to CSRINIT
\\end{verbatim}

