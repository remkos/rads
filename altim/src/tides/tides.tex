\documentclass[a4paper]{article}
\usepackage{a4wide,verbatim,som}
\title{TIDES \\ Tide Generating Routines \\
User's Guide and Reference}
\author{Remko Scharroo}
%%%%%%%%%%%%%%%%%%%%%%%%%%%%%%%%%%%%%%%%%%%%%%%%%%%%%%%%%%%%%%%%%%%%%%%%
\def\SECTION#1{\subsection{#1}}
\def\SUBSECTION#1{\subsubsection{#1}}
\def\Input#1{\vbox{\input{#1}}}
%\let\Input\input
%%%%%%%%%%%%%%%%%%%%%%%%%%%%%%%%%%%%%%%%%%%%%%%%%%%%%%%%%%%%%%%%%%%%%%%%
\begin{document}
\tableofcontents

\section{Introduction}
TIDES is a FORTRAN subroutine library that contains some tide generating
routines.

The TIDES library has been ported to various systems.
Presently it is available at DEOS/LR on the IBM Risk 6000 and the SGI
servers. At NOAA it is available for Linux.

The library can be linked to any of your
FORTRAN programs by specifying the library in you FORTRAN link step, \eg
\begin{verbatim}
f77 example.f -o example $ALTIM/lib/tides.a
\end{verbatim}

The TIDES library contains drivers for several routines written by other
institutes. The drivers are similar in definition of their calls in order
to improve exchangability.

\section{Subroutine Synopsis}
\Input{airtide}
\Input{csrtide}
\Input{etide_ce}
\Input{festide}
\Input{gottide}
\Input{webtide}
\Input{lpetide}
\Input{poletide}
\end{document}
