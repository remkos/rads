\documentstyle[european,som]{article}
\title{PIM reference and manual \\ DRAFT}
\author{Remko Scharroo}

\def\PIM{{\sc pim}}
\def\pim{\protect\ppim}
\def\ppim#1#2#3{{{\small\bf\uppercase{#1}}\bf#2\em#3}}

\begin{document}
\maketitle

\section{Introduction}
{\bf Who's \PIM?}
\begin{itemize}
\item \PIM\ is the Perfect Image Mapper, a program by Remko Scharroo and Edwin
Wisse of the Delft University of Technology's Section Space Research and
Technology.

\item \PIM\ is build on the PMPLOT subroutine library by Remko Scharroo, which
is an adjusted version of PGPLOT by Tim Pearson of CalTech.

\item \PIM\ is written to generate contour plots and colour plots of
grids of geographical data, such as sea surface height.

\item \PIM\ supports all devices that are in the current PMPLOT library
release: PostScript (colour and greyscale), XWindows, VT100, GIF, and PPM.

\item \PIM\ is fast and versatile.

\item \PIM\ uses a handy ``input card format''
\end{itemize}

\section{Command Reference}

\subsection{\pim{gridarea}{ lon0 lon1 lat0 lat1}{}}
The \pim{gridarea}{}{} command specifies the longitude and latitude
boundaries of the grids that are to be plotted.
If the card is omitted, an attempt is made to read the areas from the
grids. If the area is not given in the grid header, or when this conflicts
with the area given on the \pim{gridarea}{}{} card, an error message is
given and the program will halt. The arguments \pim{}{ lon0 lon1 lat0
lat1}{} area manditory and specify the most western and eastern longitude
and the most southern and northern longitude respectively. Negative numbers
indicate west or south, positive numbers east or north.

\subsection{\pim{surface}{ filename}{ min max}}
To plot a grid in a scale of colours, use the \pim{surface}{}{}
command. This card specifies the name of the grid file to be plotted
(\pim{}{filename}{}, manditory), and the range of values that are to
be assigned colours, \pim{}{}{min max}. If values in the grid are
less than \pim{}{}{min}, they will obtain the same colour as \pim{}{}{min},
and similar for \pim{}{}{max}.
Illegal values will always be assigned the ``bad data'' colour (index 3).

If the extreems are omitted, the minimum and
maximum value in the grid are assumed.
If the card is omitted, no grid is plotted.

It is also possible to replace the \pim{}{filename}{} by an arithmatic
expression with file names, \eg, {\tt seaheight.grd + geoid.grd}. Adding
(~+~) and subtracting (~-~) is allowed.

\subsection{\pim{illum}{}{ filename min max angle}}
To illuminate the surface that you have plotted with the \pim{surface}{}{}
command, you can use the \pim{illum}{}{} card. Specify the name of the grid
to be illuminated with the optional argument \pim{}{}{filename}. To make it
easy, if you do not specify \pim{}{}{filename}, or type a ``='', \PIM\ will
assume the same file name as on the \pim{surface}{}{} card. The optional
arguments \pim{}{}{min max}, specify the slopes that acquire the minimum
and maximum shading; \pim{}{}{angle} indicates the direction of the
simulated light source illuminating the surface, measured counter-clockwise
starting from the right. The default is illumination from the top left
corner, \ie, \pim{}{}{angle} = 135.

To provide the illumination effect, \PIM\ will automatically generate a
new colour map from the original one specified on the \pim{colour}{}{}
card. Depending on the device capabilities, each original colour will be
represented by a number of ``shades''. (See also \pim{quant}{}{} and
\pim{colour}{}{}.)

\subsection{\pim{contour}{}{ filename min max int}}
A grid may also be presented as contours, using the \pim{contour}{}{}
card. If no arguments are given, the \pim{filename}{}{} is assumed equal
to the one on the \pim{surface}{}{} card, and contours are plotted are
printed in between the different colour levels. However, the minimum
and maximum contour level, and the interval may also be specified
by the \pim{}{}{min max int} parameters.

\subsection{\pim{dither}{}{ dither1 dither2}}
To adjust the dithering of colours and shades, the \pim{dither}{}{} command
can be used. The default (if the card is omitted or used without arguments)
is the no dithering is applied between different colours, but is applied
between different shades of the same colour, \ie, \pim{}{}{dither1 dither2}
= F T. Use, for instance \pim{dither}{ T}{} to switch on the dithering
between th colours to have a smooth colour range.

\subsection{\pim{device}{ name}{}}
The card is manditory if you want to see something plotted. The
\pim{}{name}{} field on the \pim{device}{}{} card should be one of the
device names or types in the PGPLOT syntax. For instance, use
{\tt/xw} for Xwindows, {\tt plot.gif/gif} to save the plot as a GIF file
{\tt plot.gif}. Other device types are: /ps, /vps, /cps, /cvps, /ppm, /tek.

\subsection{\pim{colour}{ name}{}}
The \pim{colour}{}{} command specifies the colour map for plotting.
The \pim{}{name}{} refers to one of the colour maps in the directory
{\tt /home/edwin/coulourmaps} or one in your local directory. If the
card is omitted, a default map is assumed.

If illumination is performed, the colour range specified in the colour map
is adjusted to be two-dimensional. One dimension (the colour dimension)
will represent the original colours; the second dimension (the shade
dimension) will simulate the illumination by setting a number of shades
(dark to light) to each original colour. In total, the number of
created indices should not exceed the device capabilities. For instance,
if you use {\tt twelve.col} and
the Xwindow driver which supports 145 colours, then 16 are assigned colours,
such as the foreground and background colour, and the remaining are devided
into 12 colours with 10 shades each.
The PPM and PS drivers support ``true colour'' capabilities. For these
devices, the number of shades is set to 256.

The \pim{quant}{}{} card can be used to decrease the number of colours that
is allowed, below the device capabilities.

\subsection{\pim{quant}{}{ colours}}
To quantisize the number of allowed colours below that of the device
capabilities, use the \pim{quant}{}{} card. If the card or
the \pim{}{}{colours} argument is omitted, the maximum number of colours
allowed by the driver are assumed.

\subsection{\pim{background}{}{ filename}}
To adjust the background from being in one colour, use the
\pim{background}{}{} command. The \pim{}{}{filename} should specify a PGM
file. If the argument is omitted the TU Delft logo is plotted.

This command is only allowed for pixel devices.

(This command is not yet fully supported.)

\subsection{\pim{logo}{}{ filename x0 y0}}
To plot the TU Delft logo at the bottom left corner of the background,
use the \pim{logo}{}{} command without arguments. The arguments
\pim{}{}{filename x0 y0} may be used to specify a different PGM file or
a different location (in normalised viewport coordinates).

This command is only allowed for pixel devices.

(This command is not yet fully supported.)

\subsection{\pim{project}{ type}{ scale}}
The projection type is defined with the \pim{project}{}{} command.
Valid projection types are given in the PMPLOT manual.
The default projection is equi-rectangular (type 1).

The scale of the
plot can be given as well. By default the plot will be scaled to fit the
largest possible space in the viewport.

\subsection{\pim{viewport}{ x0 x1 y0 y1}{}}
To adjust the amount of space used by the plot, the viewport can be
adjusted with the \pim{viewport}{}{} command. The default values of the
viewport boundaries \pim{}{x0 x1 y0 y1}{} are 0.05 0.95 0.05 0.95.
Other possibilities are \pim{viewport}{ big}{} (default) or
\pim{viewport}{ fullscreen}{} or \pim{viewport}{ small}{}.

\subsection{\pim{title}{ string}{}}
The title of the plot can be specified with the \pim{title}{}{} command.
If this command is used, extra space is reserved at the bottom of the
viewport to fit the text.

\subsection{\pim{plotarea}{ lon0 lon1 lat0 lat1}{}}
The \pim{plotarea}{}{} command specifies the longitude and latitude
boundaries of the region that has to be plotted, which does not have to be
the same as the region defined by the grid boundaries.

If the card is omitted, the area that is plotted is the same as the
gridarea defined by the \pim{gridarea}{}{} command or the grid file
header.
The arguments \pim{}{ lon0 lon1 lat0
lat1}{} are manditory and specify the most western and eastern longitude
and the most southern and northern longitude of the plot respectively.
Negative numbers
indicate west or south, positive numbers east or north.

\subsection{\pim{legend}{}{ units hmin hmax hint nsubs}}
With the \pim{legend}{}{} command the plotting of a legend bar is invoked.
Automatically space is reserved on the right side or bottom of the viewport
to fit the bar. Whether the bar appears on the right or bottom depends on
the shape of the plot.

By default, the string ``meters'' is printed to indicate the units.
However, this mat be overruled by specifying the \pim{}{}{units}
parameter. In addition, the range of the bar, specified on the
\pim{surface}{}{} card can be overruled by the parameters
\pim{}{}{hmin hmax}. If you do not like the PGPLOT standard interval
generation, redefine this with \pim{}{}{hint nsubs}.

For example, if you have a grid containing heights in meters, and specified
\pim{}{}{min max} as 0.0 0.20 on the \pim{surface}{}{} card, you may
like to use \pim{legend}{ cm 0 20}{} to get the bar in centimeters.

\subsection{\pim{inter}{}{ xcells ycells}}
Normally, \PIM\ does not interpolate your grid, but this is very much
recommended, especially for pixel devices, such as /xw and /gif.
Interpolating can be invoked with the \pim{inter}{}{} command.
If this command is used, \PIM\ will interpolate the grid to pixel level.
Otherwise, your grid will be plotted as rectangular cells of the original
size of the grid cells.

If you use a PostScript device, \pim{inter}{}{} can be used to reduce
the resolution of the grid to reduce the plot size. Resolutions between
200 200 and 400 400 are optimal for these devices.

\subsection{\pim{info}{}{}}
To get some information on the grids and some options, use the
\pim{info}{}{} command.

\subsection{\pim{notation}{ filename}{}}
Use or generate a notation file.

(This command is not properly supported yet).

\subsection{\pim{box}{}{ linewidth 'xopt' xint nxsub 'yopt' yint
nysub}}
To change the default parameters for plotting of the box around the
plot surface (1 'bcnsti' 0 0 'bcnstiv' 0 0), the \pim{box}{}{} command
can be used. If the linewidth is set to 0, no box will be plotted.
Also, parts of the box may be removed by omitting some of the
'xopt' or 'yopt' parameters (have to be within quotes). The
parameters \pim{}{}{xint nxsub yint nysub} handle the printing of major
and subticks.

\subsection{\pim{bathym}{}{ filename linewidth levels}}
To plot bathymetric contours, use the \pim{bathym}{}{} command.
The argument \pim{}{}{filename} specifies the WDB data set
(default 2.nat.bth), \pim{}{}{linewidth} the line width (default 1),
and \pim{}{}{levels} up to 5 levels to be plotted (default -3500 -2000).

This card may be repeated to use more levels or different WDB data sets.

\subsection{\pim{land}{}{ filename}}
To draw a land mask the \pim{land}{}{} command is invoked. By default,
the land mask is read from the 1.lnd WDB data set, but you can
specify others with the optional \pim{}{}{filename} parameter.

This card may be repeated to use different WDB data sets.

\subsection{\pim{coast}{}{ filename linewidth}}
The \pim{coast}{}{} command controls plotting of a coast line. By default,
the 1.cil WDB data set is used, but you can
specify others with the optional \pim{}{}{filename} parameter.
In addition, the linewidth (default 1) can be redefined by the
\pim{}{}{linewidth} parameter.

This card may be repeated to use different WDB data sets.

\subsection{\pim{xgf}{ filename}{ colour marker linewidth size}}
If you want to plot markers, store the locations in XGF format and
use the \pim{XGF}{}{} command in \PIM. The parameter \pim{}{filename}{}
is manditory and specifies the XGF filename. Additional parameters
\pim{}{}{colour marker linewidth size} control the colour index, marker
type, line width, and size (in character heights).

If a color index of -999 is specified, the colour index will change
according to the height field in the XGF file and the colour range.
If a marker
type of -1 is specified you will get point markers, and -999 means:
get the marker type from the sigma field in the XGF.
Defaults for the optional arguments are -999 -1 2 1.

\end{document}
