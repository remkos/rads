\documentclass[a4paper]{article}
\usepackage{a4wide,verbatim,som}
\title{RSSUBS \\ A Whole Bunch of Routines \\
User's Guide and Reference}
\author{Remko Scharroo}
%%%%%%%%%%%%%%%%%%%%%%%%%%%%%%%%%%%%%%%%%%%%%%%%%%%%%%%%%%%%%%%%%%%%%%%%
\def\SECTION#1{\subsection{#1}}
\def\SUBSECTION#1{\subsubsection{#1}}
\def\Input#1{\vbox{\input{#1}}}
%\let\Input\input
%%%%%%%%%%%%%%%%%%%%%%%%%%%%%%%%%%%%%%%%%%%%%%%%%%%%%%%%%%%%%%%%%%%%%%%%
\begin{document}
\tableofcontents

\section{Introduction}
RSSUBS is a FORTRAN subroutine library that contains some handy routines to be
used for many purposes.

The RSSUBS library has been ported to various systems.
Presently it is available at DEOS/LR on the IBM Risk 6000 and the SGI
servers. At NOAA it is available for Linux.

The library can be linked to any of your
FORTRAN programs by specifying the library in you FORTRAN link step, \eg
\begin{verbatim}
f77 example.f -o example $ALTIM/lib/rssubs.a
\end{verbatim}

The RSSUBS library is courtesy of Remko Scharroo, Delft University of
Technology, Delft Institute for Earth-Oriented Space Research.
(E-mail: {\tt remko@deos.tudelft.nl})

\section{Subroutine Synopsis}
\Input{altbias}
\Input{anmtot}
\Input{anttom}
\Input{bubble}
\Input{carpol}
\Input{chartrans}
\Input{checkenv}
\Input{chrdat}
\Input{chrloc}
\input{covar}
\input{datearg}
\Input{dhellips}
\Input{dqsort}
\Input{dyntopo}
\Input{earth}
\Input{elevec}
\Input{epharg}
\Input{ephpos}
\Input{f1f2}
\input{fastio}
\Input{fdate}
\Input{fin}
\Input{freeunit}
\Input{gauss1d}
\Input{geocen}
\Input{geodet}
\Input{geoutm}
\Input{geoxyz}
\input{getorb}
\Input{globpres}
\Input{grdate}
\Input{helmert1}
\Input{helmert2}
\Input{i2swap}
\Input{i4swap}
\Input{intab2}
\Input{intab8}
\Input{inter8}
\Input{iqsort}
\Input{isnan}
\input{j2000}
\Input{land}
\Input{listargs}
\Input{loadcoord}
\Input{lowercase}
\Input{ltlend}
\Input{mallocf}
\Input{matsy1}
\Input{memloc}
\Input{mcw}
\Input{mdate}
\Input{meteo_subs}
\Input{mjdate}
\input{nf_subs}
\Input{nff}
\Input{noise}
\Input{nuvel1a}
\Input{odrinfo}
\Input{polcar}
\Input{quaint}
\Input{regres}
\Input{rotate}
\Input{round}
\Input{scaprd}
\Input{sfdist}
\Input{sec85}
\Input{sland}
\Input{statbar}
\Input{statinfo}
\Input{statis}
\input{strfepoch}
\input{strpepoch}
\Input{uppercase}
\Input{vecele}
\Input{vecnrm}
\Input{vecprd}
\Input{vnorm}
\Input{xyzgeo}
\Input{ymdhms}
\end{document}
